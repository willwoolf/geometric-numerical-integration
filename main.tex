\documentclass{report}

\usepackage{amsmath}
\usepackage{amssymb}
\usepackage{amsthm}


\newtheorem{theorem}{Theorem}[subsection]
\newtheoremstyle{exampstyle}
  {\topsep} % Space above
  {\topsep} % Space below
  {} % Body font
  {} % Indent amount
  {\bfseries} % Theorem head font
  {.} % Punctuation after theorem head
  {.5em} % Space after theorem head
  {} % Theorem head spec (can be left empty, meaning `normal')
\theoremstyle{exampstyle} \newtheorem{example}[theorem]{Example}
\theoremstyle{exampstyle} \newtheorem{remark}[theorem]{Remark}
\theoremstyle{exampstyle} \newtheorem{definition}[theorem]{Definition}
\theoremstyle{exampstyle} \newtheorem{lemma}[theorem]{Lemma}

\usepackage[english]{babel}
\usepackage{graphicx}
\usepackage{verbatim}
\usepackage{listings}
\usepackage{float}
\usepackage{dsfont}
\usepackage[width=6in, height=8in]{geometry}
\usepackage{xcolor}

\usepackage[
backend=biber,
natbib=true,
url=false, 
doi=true,
eprint=false
]{biblatex}
\addbibresource{sources.bib}

\definecolor{codegreen}{rgb}{0,0.6,0}
\definecolor{codegray}{rgb}{0.5,0.5,0.5}
\definecolor{codepurple}{rgb}{0.58,0,0.82}
\definecolor{backcolour}{rgb}{0.95,0.95,0.92}

\lstdefinestyle{mystyle}{
	backgroundcolor=\color{backcolour},   
	commentstyle=\color{codegreen},
	keywordstyle=\color{magenta},
	numberstyle=\tiny\color{codegray},
	stringstyle=\color{codepurple},
	basicstyle=\ttfamily\footnotesize,
	breakatwhitespace=false,         
	breaklines=true,                 
	captionpos=b,                    
	keepspaces=true,                 
	numbers=left,                    
	numbersep=5pt,                  
	showspaces=false,                
	showstringspaces=false,
	showtabs=false,                  
	tabsize=2
}

\lstset{style=mystyle}

\begin{document}

\chapter{Hamiltonian Mechanics and Symplecticity}

\section{Definitions}

\subsection{Principles of Hamiltonian Mechanics}

The most general way to define a dynamical system is to denote position by the vector $\mathbf{x}$ and express the system as
\begin{equation}
	\mathbf{\dot{x}} = f(\mathbf{x})
	\label{eqn:dyn}
\end{equation}
for an arbitrary function $f$, using dot notation for time derivative.
This can express any dynamical system from reduction to a first order vector equation.

For a Hamiltonian system in one dimension, we denote position by the variable $q$, and momentum by $p$.
The Hamiltonian $H(q,p)$ is a first integral of the system, used to denote the total energy.
In this case, we write the system as
\begin{equation}
	\dfrac{\mathrm{d}}{\mathrm{d}t}\begin{pmatrix}
		q \\
		p
	\end{pmatrix} = \begin{bmatrix}
		\strut 0 & \strut 1 \\
		\strut -1 & \strut 0
	\end{bmatrix}\begin{bmatrix}
	\dfrac{\strut\mathrm{d}H}{\strut\mathrm{d}q} \\
	\dfrac{\strut\mathrm{d}H}{\strut\mathrm{d}p}
	\end{bmatrix}.
\end{equation}
Explicitly this expands as
\begin{align*}
	&\dot{q} = \dfrac{\partial H}{\partial p}
	&
	\dot{p} = \dfrac{-\partial H}{\partial q}&.	
\end{align*}
If we write $\mathbf{x} = (q,p)^\mathrm{T}$ and denote the matrix as $\mathbf{J}$, which we will refer to as the symplectic matrix,
then the statement of a hamiltonian system is of the form 
\begin{equation}
	\mathbf{\dot{x}} = \mathbf{J}\nabla H(\mathbf{x}),
\end{equation}
which resembles Equation \ref{eqn:dyn}.

For an $n$-dimensional problem,
we instead denote the variables $q_1, \mathellipsis, q_n, p_1, \mathellipsis, p_n$ as the position and momentum in the directions of each basis vector respectively.
Write $\mathbf{x} = (q_1, \mathellipsis, q_n, p_1, \mathellipsis, p_n)$.
Define $\mathbf{J}$ to be
\begin{equation}
	\mathbf{J} = \begin{bmatrix}
		\mathbf{0}_n & \mathbf{I}_n \\
		-\mathbf{I}_n & \mathbf{0}_n
	\end{bmatrix}
\end{equation}
and the problem can be written as $\mathbf{\dot{x}} = \mathbf{J}\nabla H(\mathbf{x})$.
We often call $\mathbf{J}$ the symplectic matrix.

Define the flow $\varphi_t$ as a map of the problem from an initial time to the time $t$.
Specifically, we write $\varphi_t(\alpha) = \mathbf{x}(t)$ given the initial condition $\mathbf{x}(t=0) = \alpha$. Hence the flow $\varphi_t:\mathds{R}^{2n}\rightarrow \mathds{R}^{2n}$ is a map from the initial point of the system to the time $t$.

The Jacobian matrix $\varphi'_t(\mathbf{\alpha})$ is often called the sensitivity of the flow.
We show the element-wise expression:
\begin{align*}
	\left. \varphi'_t(\alpha) \right)_i &= \left. \frac{\partial}{\partial \alpha} \varphi_t(\alpha) \right)_i \\
	&= \left. \frac{\partial}{\partial \alpha} x_i(t) \right)_i \\
	&= \begin{bmatrix}
		\frac{\partial x_i}{\partial \alpha_1} & \mathellipsis & \frac{\partial x_i}{\partial \alpha_{2n}}
	\end{bmatrix},
\end{align*}
which is the Jacobian.
A flow is symplectic if it satisfies $\varphi'_t(\alpha)^\mathrm{T} \mathbf{J} \varphi'_t(\alpha) = \mathbf{J}.$

\section{Examples}
\subsection{Forward Euler Method}

If a problem has a closed-form solution, then we can find an explicit expression for the flow.
However, many problems must instead be solved numerically, often using a finite difference method.
We will focus on symplectic and non-symplectic methods for solving differential equations.
If a numerical method maintains symplecticity, it is a symplectic integrator.
These are a class of geometric numerical integrators.

Given an ODE $\dot{\mathbf{x}} = f(\mathbf{x})$, a numerical method is an iteration of the form $\mathbf{x}_{n+1} = \mathbf{x}_n + F(t_n, \mathbf{x}_n; h)$
We can write a step as the map $\Phi: \mathbf{x}_n \rightarrow \mathbf{x}_{n+1}$. The map $\Phi$ is the numerical flow:
rather than mapping from the initial condition to a point at time $t$, we map the solution from one point $t_n$ to the next in a discretised time interval.

Consider the forward Euler method: for $\dot{\mathbf{x}} = f(\mathbf{x})$, the iteration is $\mathbf{x}_{n+1} = \mathbf{x}_n + hf(\mathbf{x}_n)$.
We can apply the Euler method to examples in order to evaluate its properties.

First, we look at the simple harmonic oscillator $m \ddot{x} = -k x$.
In Hamiltonian variables this is
\begin{equation*}
	\begin{aligned}
		&\dot{q} = \frac{p}{m}, &\dot{p} = m\ddot{q} = -kq.
	\end{aligned}
\end{equation*}
A suitable Hamiltonian is
\begin{equation*}
	H = \frac{1}{2m}p^2 + \frac{1}{2}kq^2.
\end{equation*}
Applying the forward Euler method to the problem, we get
\begin{equation*}
	\begin{aligned}
		q_{n+1} &= q_n + h\frac{\partial H}{\partial p}(q_n, p_n) &= q_n + \frac{p_n}{m}\\
		p_{n+1} &= p_n - h\frac{\partial H}{\partial q}(q_n, p_n) &= p_n - kq_n
	\end{aligned}
\end{equation*}
which can be written as a matrix equation
\begin{equation*}
	\begin{pmatrix}
		q_{n+1} \\
		p_{n+1}
	\end{pmatrix} = \begin{bmatrix}
		1 & 1/m \\
		-k & 1
	\end{bmatrix} \begin{pmatrix}
		q_n \\
		p_n
	\end{pmatrix}.
\end{equation*}
where the matrix is the numerical flow map.
We could show that this matrix does not satisfy the symplectic identity,
but it is enough to show that its determinant is not $1$, which is a requirement for symplectic matrices.

\subsection{Example of a symplectic flow}
We can demonstrate symplecticity with the actual flow map. Note the general closed form solution:
\begin{equation*}
	\begin{aligned}
		q(t) &= A\cos\left( \sqrt{\frac{k}{m}} t \right) + B \sin\left( \sqrt{\frac{k}{m}}t \right) \\
		p(t) &= B\sqrt{km} \cos\left( \sqrt{\frac{k}{m}} t \right) -A\sqrt{km} \sin\left( \sqrt{\frac{k}{m}}t \right) 
	\end{aligned}
\end{equation*}
for arbitrary constants $A$ and $B$.
Impose initial conditions:
suppose $q(t=0) = q_0,~ p(t=0) = p_0.$
Define $\omega = \sqrt{k/m}$ for shorthand, and apply the initial conditions to find that the coefficients are $A = q_0,~ B = p_0/m \omega$
Hence the particular solution is
\begin{equation*}
	\begin{aligned}
		q(t) &= q_0 \cos\left( \omega t \right) + \frac{p_0}{m\omega} \sin\left( \omega t \right) \\
		p(t) &= p_0 \cos\left( \omega t \right) - q_0 m\omega \sin\left( \omega t \right).
	\end{aligned}
\end{equation*}
We are in the position to evaluate the Jacobian entry-wise.
\begin{equation*}
	\varphi'_t \begin{pmatrix}
		q_0 \\
		p_0
	\end{pmatrix} = \begin{bmatrix}
		\frac{\partial q}{\partial q_0} & \frac{\partial q}{\partial p_0} \\
		\frac{\partial p}{\partial q_0} & \frac{\partial p}{\partial p_0}
	\end{bmatrix} = \begin{bmatrix}
		\cos(\omega t) & \frac{1}{m\omega} \sin(\omega t) \\
		-m\omega \sin(\omega t) & \cos(\omega t)
	\end{bmatrix},
\end{equation*}
and if we now plug this into the symplectic identity we get
\begin{align*}
	\varphi'_t(x_0)^\mathrm{T} \mathbf{J} \varphi'_t(x_0) &= \begin{bmatrix}
		\cos(\omega t) & -m\omega \sin(\omega t) \\
		\frac{1}{m\omega} \sin(\omega t) & \cos(\omega t)
	\end{bmatrix} \begin{bmatrix}
		0 & 1 \\
		-1 & 0
	\end{bmatrix} \begin{bmatrix}
		\cos(\omega t) & \frac{1}{m\omega} \sin(\omega t) \\
		-m\omega \sin(\omega t) & \cos(\omega t)
	\end{bmatrix} \\
	&= \begin{bmatrix}
		m \omega \sin(\omega t) & \cos(\omega t) \\
		-\cos(\omega t) & \frac{1}{m \omega} \sin(\omega t)
	\end{bmatrix} \begin{bmatrix}
		\cos(\omega t) & \frac{1}{m\omega} \sin(\omega t) \\
		-m\omega \sin(\omega t) & \cos(\omega t)
	\end{bmatrix} \\
	&= \begin{bmatrix}
		m \omega \sin(\omega t) \cos(\omega t) - m \omega \sin(\omega t) \cos(\omega t)  & \sin^2(\omega t) + \cos^2(\omega t) \\
		-\cos^2(\omega t) - \sin^2(\omega t) & -\frac{1}{m\omega}\cos(\omega t)\sin(\omega t) + \frac{1}{m \omega}\cos(\omega t)\sin(\omega t)
	\end{bmatrix} \\
	&= \begin{bmatrix}
		0 & 1 \\
		-1 & 0
	\end{bmatrix} = \mathbf{J}.
\end{align*}
Hence the flow is symplectic.

\subsection{A symplectic integrator}
With symplectic integration, we are interested in numerical methods for which the numerical flow also attains symplecticity.
One such example is the symplectic Euler method
\begin{equation*}
	\begin{pmatrix}
		q_{n+1} \\
		p_{n+1} 
	\end{pmatrix} = \begin{pmatrix}
		q_{n} \\
		p_{n}
	\end{pmatrix} + h \nabla H(q_{n}, p_{n+1}).
\end{equation*}
For an arbitrary problem this method is implicit. However, many problems have a separate Hamiltonian that allows the iteration to be performed in two explicit steps. 
The simple harmonic oscillator has a Hamiltonian $H(q, p) = \frac{1}{2m}p^2 + \frac{1}{2}kq^2$ which can be written as $H(q, p) = V(q) + T(p)$, where $V(q)$ represents kinetic energy and $T(p)$ represents potential energy of the system.
We expand out the method as
\begin{equation*}
	\begin{pmatrix}
		q_{n+1} \\
		p_{n+1} 
	\end{pmatrix} = \begin{pmatrix}
		q_{n} \\
		p_{n}
	\end{pmatrix} + h \mathbf{J} \begin{pmatrix}
		V'(q_n) \\
		T'(p_{n+1})
	\end{pmatrix} = \begin{pmatrix}
		q_{n} + \frac{h}{m}p_{n+1} \\
		p_{n} - hk q_n
	\end{pmatrix}.
\end{equation*}
On inspection, we can perform this iteration separably: compute $p_{n+1}$ using $p_n$ and $q_n$,
then use $q_n$ and $p_{n+1}$ to compute $q_{n+1}$.
Important to note is that this is specifically the symplectic Euler-VT method, since we evaluate $V'$ before $T'$,
and the symplectic Euler-TV method is an alternative which computes in the opposite order analogously.
For simplicity we will stick with the VT method for examples.
We find an expression for the numerical flow:
\begin{equation*}
	\begin{aligned}
		\begin{pmatrix}
			q_{n+1} \\
			p_{n+1} 
		\end{pmatrix} &= \begin{pmatrix}
			q_{n} + \frac{h}{m} \left( p_{n} - h \omega q_n \right) \\
			p_{n} - h \omega q_n
		\end{pmatrix} \\
		&= \begin{bmatrix}
			1 - \frac{h^2 k}{m} & \frac{h}{m} \\
			-hk & 1
		\end{bmatrix} \begin{pmatrix}
			q_n \\
			p_n
		\end{pmatrix} \equiv \Phi \begin{pmatrix}
			q_n \\
			p_n
		\end{pmatrix}
	\end{aligned}
\end{equation*}
and just like before, test the symplectic identity
\begin{equation*}
	\Phi^\mathrm{T}\mathbf{J}\Phi = \begin{bmatrix}
		\left(1-\frac{h^2 k}{m}\right)(-hk) + \left(\frac{h^2 k}{m}-1\right)(-hk) & 1 - \frac{h^2 k}{m} + \frac{h^2 k}{m} \\
		-\frac{h^2 k}{m} + \frac{h^2 k}{m} -1 & \frac{h}{m} - \frac{h}{m}
	\end{bmatrix} = \begin{bmatrix}
		0 & 1 \\
		-1 & 0
	\end{bmatrix} = \mathbf{J}.
\end{equation*}
Hence this method is symplectic for this example.

\section{Numerical and Symplectic Integrators}

It is important to note at the moment that the only example we have looked at is the simple harmonic oscillator, which has a closed form solution.
Our results so far only serve for understanding the definition of symplecticity.
Further on we will discuss and explore real applications of symplectic methods.
% go into more detail of forward euler

\subsection{Stability and Symplecticity}
In contrast to our examples so far, we usually solve an ODE numerically when we cannot find a closed form solution, and we want to produce a numerical approximation with which we can analyse its behaviour.
However, numerical integrators such as the forward Euler method do not always provide reliable solutions to the problems given.
In oder to explore this further, we need to introduce the concept of A-stability.

Consider the linear test problem $\dot{x} = \lambda x$. Given an initial condition $x(t=0) = x_0$, $x$ decays to $0$ as $t \rightarrow \infty$ if $\lambda$ is negative.
Now consider how the forward Euler approximation behaves.
We have the iteration $x_{n+1} = x_n + hf(x_n) = x_n + h(\lambda x_n) = (1 + h\lambda)x_n$.
Hence, given an initial condition, $x_n = (1 + h\lambda)^n x_0$, which decays to zero if $|1+h\lambda| < 1$.
Even if $\lambda < 0$, we can choose values of $h$ for which the sequence does not converge to zero.
A numerical method is called A-stable if, when applied to the linear test problem, the sequence converges to zero for all $h\lambda$ such that $\operatorname{Re}(h\lambda) < 0$.
Euler's method is not A-stable, making it susceptible to long timespan error growth.

We will look at the concept of the adjoint method. Given a method defined by a numerical flow $\Psi_h$,
the adjoint $\Psi*_h$ is the method that satisfies $\Psi^*_{-h} = \Psi^{-1}_h$.
In words, stepping backward with the adjoint method is equivalent to stepping forward with the inverse.
Adjoint methods are very useful. Given an arbitrary method $\Psi_h$,
the method $\Psi_{h/2} \circ \Psi_{h/2}^*$ is symmetric.
A symmetric method is a method that satisfies $\Psi_h^{-1} = \Psi_{-h}$.
This means if we integrate forwards in time by a particular step, and then integrate back,
we return to the original value since we are equivalently applying the inverse.

We can show that the implicit Euler method is the adjoint of explicit Euler.
First, let $\Psi_h^I$ denote the implicit method and let $\Psi_h^E$ denote the explicit method.
We want to show that $\Psi_h^E = (\Psi_{-h}^I)^{-1}$.
This is equivalent to $\Psi_{-h}^I \circ \Psi_h^E (x) = x$.
If we expand this composition, we obtain
\begin{align*}
	\Phi_{-h}^I \left( \Phi_j^E  (x) \right) &= \Phi_{-h}^I \left( x + h f(x) \right) \\
	&= x + h f(x) - h f\left( \Phi_{-h}^I \left( x + h f(x) \right) \right) \\
	&= x + h f(x) - h f\left( \Phi_{-h}^I \left( \Phi_h^E (x) \right) \right),
\end{align*}
and $\Phi_{-h}^I (\Phi_h^E (x)) = x$ solves this equation. Hence $(\Phi_{-h}^I)^{-1} = \Phi_h^E$.

Other methods of note are the implicit midpoint method and the trapezium method. Implicit midpoint is the method
\begin{equation*}
	\Phi_h^M (x_n) = x_{n+1} = x_n + hf\left(\frac{x_n + x_{n+1}}{2}\right).
\end{equation*}
The trapezium method is similar:
\begin{equation*}
	\Phi_j^T (x_n) = x_{n+1} = x_n + h \left(\frac{f(x_n) + f(x_{n+1})}{2}\right)
\end{equation*}
The implicit midpoint method is symplectic, but the trapezium method is not.
However, it can be shown that these are conjugate methods, meaning that they exhibit similar long-term behaviour.
Two methods $\Psi, \Phi$ are conjugate if there exists a map $\chi$ such that $\Phi = \chi^{-1} \Psi \chi$.

%% find a defined inverse so that we can show conjugacy.

% goal of symplectic - area preserving
% symplectic in hamiltonian mechanics preserving a first integral
Symplecticity is stronger than A-stability, but it is similar in how it aims to provide stable solutions for long timespan integrations.
The formal definition is that a symplectic integration method maintains the form $\mathrm{d}q_i \wedge \mathrm{d}p_i$ with $i = 1, \mathellipsis, n$ for an $n$-dimensional problem.
The form is an infinitesimal area generated by the infinitesimals in $q, p$.
For a one-dimensional problem, we can produce the phase portrait by plotting $p$ against $q$.
A single point in the phase portrait represents the state of the dynamical system at a fixed point in time.
The flow and the numerical flow are maps between points in the phase portrait.
If we think of the phase portrait as a solution space for a one-dimensional dynamical system, we can say that symplectic methods preserve the area of the solution space.
This is the paradigm of symplectic integration: by maintaining area of the phase space under mapping of the numerical flow, we maintain qualitative behaviour of the ODE over long timespans.
Furthermore, the conservation of area is equivalent to a symplectic method maintaining a first integral of the system, which is the Hamiltonian $H$.

The dilemma with symplectic integration is that it can be an extremely expensive implementation, when another method might suffice.
For applications in celestial mechanics, symplectic integrators are extremely important, since we may need to estimate the motion of celestial bodies far into the future, to a required accuracy.
However, many examples would be exhaustive for symplectic methods, since it would be more computationally efficient to use an error-controlled explicit integrator.
In this case, we accrue error over time, but we can make a restriction of the local error in order to keep the global error below a particular threshold.
An extremely popular family of integrators would be the explicit Runge-Kutta methods.

\subsection{Runge-Kutta Methods}

These are methods of the form
\begin{equation*}
	x_{n+1} = x_n + h(\sum_{i = 1}^{s} b_i k_i )
\end{equation*}
where
\begin{equation*}
	k_i = f(t_n + c_i h,~ x_n + h\sum_{j = 1}^{s} a_{ij}k_j).	
\end{equation*}
An explicit Runge Kutta method is a sequence of explicit steps to gain a better approximation.
We aim to choose the parameters such that we can reduce the order of the error for each step.
For ease of notation, we introdue the Butcher tableau for a Runge Kutta method
\begin{equation*}
	\begin{array}{c|ccc}
		c_1  &a_{11} &\hdots &a_{1s} \\
		\vdots &\vdots & &\vdots \\
		c_s &a_{s1} &\hdots &a_{ss} \\
		\hline
		 &b_1 &\hdots &b_s
	\end{array}
\end{equation*}
which also helps us regard the Runge Kutta method as defined by vectors and a matrix.

\subsection{Stability}
Consider the linear test problem $\dot{x} = \lambda x$.
We have already explored stability for the basic Euler methods.
We will look at an explicit 3-stage method.
First, find expressions for the $k_i$:
\begin{align*}
	k_1 &= \lambda x_n, \\
	k_2 &= \lambda\left( x_n + h a_{21}k_1 \right) \\
	&= \left( \lambda + h a_{21}\lambda^2 \right)x_n, \\
	k_3 &= \lambda \left( x_n + h a_{31}k_1 + h a_{32}k_2 \right) \\
	&= \lambda \left( x_n + h a_{31} \lambda x_n + h a_{32}\left( \lambda + h a_{21} \lambda^2 \right) x_n \right) \\
	&= \left( \lambda + h a_{31}\lambda^2 + h a_{32}\lambda^2 + h^2 a_{32}a_{21}\lambda^3 \right) x_n.
\end{align*}
therefore
\begin{align*}
	x_{n+1} &= x_n + h \left( b_1 k_1 + b_2 k_2 + b_3 k_3 \right) \\
	&= x_n + h b_1 \lambda x_n + h b_2 \left(\lambda + h a_{21} \lambda^2\right)x_n + hb_3 \left( \lambda + h a_{31} \lambda^2 + h a_{32} \lambda ^2 + h^2 a_{32} a_{21} \lambda^3\right)x_n \\
	&= \left(
		1 + \left( b_1 + b_2 + b_3 \right) h\lambda + \left(
			b_2 a_{21} + b_3 (a_{31} + a_{32})
		\right)h^2\lambda^2 + \left(
			b_3 a_{32} a_{21}
		\right)h^3\lambda^3
	\right)x_n.
\end{align*}
This is a polynomial on $h\lambda$, which we denote by $R(h\lambda)$.
To ensure A-stability, we require that $|R(h\lambda)| < 0$ for $h \lambda$ in the left half of the complex plane.
For any explicit method, the stability function is a polynomial.
For any polynomial $p(y)$, we diverge to infinity as $|y| \rightarrow \infty$.
Hence it is impossible for any explicit Runge-Kutta method to be A-stable.

In general, the stability function for a Runge-Kutta method is
\begin{equation*}
	r(\lambda) = \frac{\det(I - \lambda A + \lambda eb^\mathrm{T})}{\det(I - \lambda A)}
\end{equation*} %% cite this result
where $A = (a_{ij})$, $b = (b_i)$ and $e$ is the vector of all ones.
The stability function is a rational function, and is a polynomial if the concerned method is explicit.

%% more stuff go here

\subsection{Symplectic RK Methods}

We have shown that all A-stable Runge-Kutta methods are necessarily implicit.
This is also the case for all symplectic Runge-Kutta methods.

% one theorem is that if the method satisfies the implicit criterion then it preserves quadratic integrals
% another result shows that RK methods that preserve quadratic integrals are


\section{Symplecticity of a Hamiltonian System}

Earlier, we looked at the symplecticity of the flow of a particular Hamiltonian system.
We will now generalise this result to any Hamiltonian system.
It should be noted that the symplecticity of the flow is equivalent to the system itself being Hamiltonian.

\begin{theorem}
	A dynamical system is Hamiltonian if and only if its flow is symplectic.
\end{theorem}
\begin{proof}
$(\Rightarrow)$ Consider the flow at time $t=0$. By definition, $\varphi_t(x_0) = x(t)$ given $x(0) = x_0$,
therefore $\varphi_0(x_0) = x_0$. Hence the sensitivity at $t=0$ is $\varphi'_0(x_0) = I$ the identity matrix.
The symplectic identity is satisfied trivially: $(\varphi'_0(x_0))^\mathrm{T} J \varphi'_0(x_0) = IJI = J$.

Now, instead of finding an expression of the symplectic identity at time $t$,
we show that this quantity is unchanging in time.
By differentiating, the expression distributes by the product rule:
\begin{equation*}
	\frac{\mathrm{d}}{\mathrm{d}t} \left(
		\varphi'_t(x_0)^\mathrm{T} J \varphi'_t(x_0)
	\right) = \left(
		\frac{\mathrm{d}}{\mathrm{d}t} \varphi'_t(x_0)^\mathrm{T}
	\right) J \varphi'_t(x_0) + \varphi'_t(x_0)^\mathrm{T} J \left(
		\frac{\mathrm{d}}{\mathrm{d}t} \varphi'_t(x_0)
	\right).
\end{equation*}
We can find an expression for the derivative term:
\begin{align*}
	\frac{\mathrm{d}}{\mathrm{d}t} \varphi'_t(x_0) &= \frac{\mathrm{d}}{\mathrm{d}t} J \nabla H(\varphi_t(x_0)) \\
	&= J \nabla^2 H(\varphi_t(x_0)) \varphi'_t(x_0).
\end{align*}
Now plug this back in, becoming
\begin{align*}
	\frac{\mathrm{d}}{\mathrm{d}t} \left(
		\varphi'_t(x_0)^\mathrm{T} J \varphi'_t(x_0)
	\right) &= \left( J \nabla^2 H(\varphi_t)\varphi'_t \right)^\mathrm{T} J \varphi'_t 
	+ \varphi'_t J \left( 
		J \nabla^2 H(\varphi_t) \varphi'_t
  	\right) \\
	&= (\varphi'_t)^\mathrm{T} \nabla^2 H(\varphi_t)^\mathrm{T} J^\mathrm{T} J \varphi'_t
	+ (\varphi'_t)^\mathrm{T} J^2 \nabla^2 H(\varphi_t) \varphi'_t(x_0) \\
	&= (\varphi'_t)^\mathrm{T} \nabla^2 H(\varphi_t)^\mathrm{T} \varphi'_t - (\varphi'_t)^\mathrm{T} \nabla^2 H(\varphi_t) \varphi'_t
\end{align*}
since $J^\mathrm{T}J = I$ and $J^2 = -I$.
Under the assumption that the Hessian matrix $\nabla^2 H(\varphi_t)$ is symmetric,
this expression evaluates to zero and hence the symplectic identity is satisfied for all $t$ and we are done.

$(\Leftarrow)$ Assuming that the sensitivity satisfies the symplectic identity, we want to show that the system is Hamiltonian.

\end{proof}


% properties required for A-stability
% properties required for symplecticity

% error control on explicit method versus symplectic method - what is the cost


%% symplecticity proof with differentiation




% symplectic RK methods preserving quadratic first integrals
% the commutative diagram result




\appendix












\end{document}